\documentclass{mlthesis}
% put your own "\usepackage" here:
%\usepackage{graphicx}

% uncomment the correct line
\thesistype{BSc thesis}   
\thesistype{MSc thesis}   
\thesistype{PhD thesis}

% put your name ant the title and the date of the thesis here:
\author{No Name}
\title{Cool thesis title}
\date{1. February 2018}  

\begin{document}

% create cover page
\maketitle
\begin{abstract}
  Lorem ipsum dolor sit amet, consetetur sadipscing elitr, sed diam
  nonumy eirmod tempor invidunt ut labore et dolore magna aliquyam
  erat, sed diam voluptua. At vero eos et accusam et justo duo dolores
  et ea rebum. Stet clita kasd gubergren, no sea takimata sanctus est
  Lorem ipsum dolor sit amet. Lorem ipsum dolor sit amet, consetetur
  sadipscing elitr, sed diam nonumy eirmod tempor invidunt ut labore et
  dolore magna aliquyam erat, sed diam voluptua. At vero eos et accusam
  et justo duo dolores et ea rebum. Stet clita kasd gubergren, no sea
  takimata sanctus est Lorem ipsum dolor sit amet.
\end{abstract}

\begin{zusammenfassung}
  Und das ganze noch einmal auf deutsch.    Lorem ipsum dolor sit amet, consetetur sadipscing elitr, sed diam
  nonumy eirmod tempor invidunt ut labore et dolore magna aliquyam
  erat, sed diam voluptua. At vero eos et accusam et justo duo dolores
  et ea rebum. Stet clita kasd gubergren, no sea takimata sanctus est
  Lorem ipsum dolor sit amet. Lorem ipsum dolor sit amet, consetetur
  sadipscing elitr, sed diam nonumy eirmod tempor invidunt ut labore et
  dolore magna aliquyam erat, sed diam voluptua. At vero eos et accusam
  et justo duo dolores et ea rebum. Stet clita kasd gubergren, no sea
  takimata sanctus est Lorem ipsum dolor sit amet.
\end{zusammenfassung}

%%put more stuff on the cover page if wanted
\begin{note}
  This is for other informaiton that should appear on the cover page
\end{note}

\newpage

\chapter{Chapters}

New chapters start a new page.

\chapter{How to write code/toolbox}

Create a toolbox.  Then you can use that toolbox inside a Jupyter
Notebook.

\chapter{Trennungen}

Many Hyphnations in Latex are wrong!  Please check them!

\chapter{How to include code?}

\chapter{How to write math?}

See also Knuth, Mathematical writing.

\chapter{Language, mistakes and dictionary}

\begin{itemize}
\item omit words like ``very'', oder ``sehr''.
\item bitte keinen Nominalstil (kein Beamtendeutsch)
\end{itemize}

\section{English mistakes}
\begin{center}
  \begin{tabular}{|l|l|}
    WRONG                       & CORRECT                     \\ \hline
   %--------------------------------------------------------------------
    2D                          &  two-dimensional            \\
    Stochastic Gradient Descent & stochastic gradient descent \\
    mixtures of gaussians       & mixtures of Gaussians 
  \end{tabular}
\end{center}

\subsection{Compare to or with}

\subsection{Which or that}
Understand the difference between restrictive and unrestrictive
clauses.  Note that only one of them requires a comma.

\subsection{Usually}

Usually, you put a comma after ``usually'' at the beginning
of the sentence.

\subsection{Capitalization}

In headlines only the first word should be capitalized, not all of the
words (or only the nouns).  Exceptions are ``Bayesian'', ``Gaussian'',
..., i.e. words derived from names.

Also note that we refer to Figure 1 but talk about figures, i.e. for a
particular figure, e.g. Figure 1, we capitalize, but talking about
some unspecified figure, we do not capitalize.

\subsection{Equation number}

Always put equation numbers, since even you might not refer to a
particular equation, your reviewer might do it.

\section{Deutsche Fehler}
\begin{center}
  \begin{tabular}{|l|l|}
    FALSCH   & RICHTIG \\
    \hline
    dem selben    & demselben \\
    Standart      & Standard  \\
  \end{tabular}
\end{center}

\section{Dictionary English --- German}
  
\begin{center}
  \begin{tabular}{|l|l|}
    \hline 
    english                            & deutsch                   \\
    \hline
    Gaussian distribution              & Gaussverteilung           \\
    KL-divergence                      & KL-Divergenz              \\
    convolution                        & Konvolution, Faltung      \\
    cumulative density function (CDF)  & Verteilungsfunktion       \\
    data point                         & Datenpunkt                \\
    data set                           & Datenmenge                \\
    decoder                            & Decoder                   \\
    deconvolution                      & Dekonvolution             \\
    deep neural net (DNN)              & tiefes neuronales Netz    \\
    encoder                            & Encoder                   \\
    epoch                              & Durchlauf                 \\
    feature space                      & Merkmalsraum              \\
    gradient descent                   & Gradientenabstieg         \\
    input space                        & Eingaberaum               \\
    latent space                       & latenter Raum             \\
    learning rate                      & Lernrate                  \\
    limit                              & Limes                     \\
    lower bound                        & untere Schranke           \\
    matrix                             & Matrix                    \\
    moving average                     & gleitender Mittelwert     \\
    neural net (DNN)                   & neuronales Netz           \\
    noise                              & Rauschen                  \\
    normal distribution                & Normalverteilung          \\
    objective function                 & Zielfunktion              \\
    preprocessing                      & Vorverarbeitung           \\
    posterior distribution             & A-posteriori-Verteilung   \\
    prior distribution                 & A-priori-Verteilung       \\
    probability density function (PDF) & Wahrscheinlichkeitsdichte \\
    signal                             & Signal                    \\
    upper bound                        & obere Schranke            \\
    vector                             & Vektor                    \\
    weights                            & Gewichte  \\
    \hline
  \end{tabular}
\end{center}

\section{Rechtschreibung in BSc-, MSc- und PhD-Arbeiten}

\begin{itemize}
\item oft falsche geschriebene Wörter:
  \begin{center}
    \begin{tabular}{|l|}
      \hline
      des Weiteren  \\
      infolgedessen   \\
      Gaussverteilung  \\\hline
    \end{tabular}
  \end{center}
\item Kommaregeln mit Infinitivkonstruktionen
\end{itemize}
\end{document}
